\title{Opening Up Official Statistics with the \CRANpkg{pxweb} Package}
\author{by Måns Magnusson, Leo Lahti}

% Other authors?
% Janne Huovari, many commits
% Markus Kainu, possibly figures & cheat sheet
% Pyry Kantanen, R-universe, technical & submission support, other

\maketitle

%An abstract of less than 150 words.
\abstract{Abstract \CRANpkg{pxweb} R package here.}

%\begin{itemize}
%    \item Motivation here. Why do we need this package? We want to use the open data pipeline, access, use and cite.
%    \item More and more data is available from National Statistical Agencies (see commented out text below).
%    \item We need tools to access the data that is simple and efficient to use (see commented out text below).
%    \item Citing data is a problem in itself, but important. It should be simplifed as much as possible.
%\end{itemize}


% Motivation here. Why do we need this package?

Open data science workflows rely heavily on algorithmic tools for data
retrieval and analysis. Automating major parts of the data science
workflow, such as finding, accessing, integrating, citing, and
reporting data, is helping the end users to dedicate more time on the
actual statistical analysis and interpretation. This can greatly
increase the value of the available data sources and facilitate
reproducible research \citep{Gandrud13, Boettiger2015} and the sharing
of findable, accessible, interoperable, and reusable (FAIR) data
\cite{xxx}. However, the tools often remain missing even when
the data is made available. For instance, whereas dozens of
statistical authorities have started to share data through the PX-WEB
API, a dedicated R package that provides a unified access to these
data collections has been missing. We introduce here the \CRANpkg{pxweb} R
package that has been designed to facilitate seamless access to open
data collections through the PC-Axis API \cite{xxx}, which is widely
adopted by national and international statistical organizations.

% More and more data is available from National Statistical Agencies

Statistical authorities are now sharing steadily increasing
collections of official statistics and other open data
resources \cite{xxx}. For instance, international agencies such as
Eurostat\footnote{\url{http://ec.europa.eu/eurostat/data/database}} \cite{xxx},
ILO\footnote{\url{https://www.ilo.org/ilostat}} \cite{xxx} ,
FAO\footnote{\url{http://www.fao.org/faostat}} \cite{xxx} and World
Bank\footnote{\url{https://data.worldbank.org}} \cite{xxx} have
released popular open data services. Altogether, national and
international statistical authorities are now sharing massive amounts
of open data on national and international aspects of politics,
economics, demography, health, infrastructure, climate, and other
areas. Such statistical data sets can be available with a great
geographical resolution, with time series spanning several years to
decades or even centuries \cite{xxx}.

% We need tools to access the data that is simple and efficient to use

Opening up official statistics is, however, only the first step
towards realizing their full potential and value. There is a clear
need for automated tools to access these data resources that are
simple and efficient to use. Dedicated software packages help to
simplify, standardize, and automate analysis workflows, taking into
account variations in raw data formats, access details, and typical
use cases so that the end users can avoid repetitive programming
tasks, avoid potential misinterpretations and coding errors, and save
time. Consequently, various developers have released software tools to
facilitate the use of statistical data resources. Instead of merely
providing tools for data browsing and standard retrieval, we emphasize
the need for algorithms that provide the seamless bridge between
original data sources and downstream analysis tasks in statistical
software languages. In R, various packages have been recently released
for generic open data retrieval, including for instance
quandl \cite{quandl} and pdfetch \cite{pdfetch}, and for more
dedicated access to specific data sources such as open data from
Eurostat \cite{Lahti17eurostat}, World Bank
(\CRANpkg{WDI}; \citealt{WDI}), Open Street Map
(\CRANpkg{osmar}; \citealt{osmar}) and other sources.

% Citing data is a problem in itself, but important. It should be
% simplifed as much as possible.

Data citations are an important but often neglected aspect of data
reuse. Guidelines for data sharing have emphasized the need to
document the specific data versions, access times, and
sources. Ideally, this information should be cited in a standardized
format. This process should be simplifed as much as possible.

% TODO
% some relevant initial references for citing data
% https://www.openoffice.org/bibliographic/bibtex-defs.pdf
% https://tex.stackexchange.com/questions/109127/how-would-i-cite-a-dataset-with-bibtex
% http://www.dcc.ac.uk/resources/how-guides/cite-datasets#fn11
% https://libguides.ub.uu.se/referensguiden/harvard\_exempel
% https://www.scb.se/Upload/PC-Axis/Download/PX-Web/2017v1/Release-notes-pxweb-2017-v1.pdf
% https://www.scb.se/sv\_/PC-Axis/Documentation/Error-codes-PC-Axis/

The \CRANpkg{pxweb} R package is addressing these needs and provides
mature and tested tools to find, access, and cite official statistics
and other information shared in the widely adopted PC-Axis format.


\subsection[PXWEB and PC-Axis]{PXWEB and PC-Axis}

\begin{itemize}
    \item The PXWEB and PC-Axis tool chains. Background. (see commented out text below)
    \item What is PXWEB and PC-Axis, where does this come from. Who uses this etc.
    \item Are there limitations in query sizes etc? Is this applicable (borrowed from eurostat manuscript): "The data sets are stored in cache by default to avoid repeated downloads of identical data and to speed up the analysis. Storing an exact copy of the retrieved raw data on the hard disk will also support reproducibility when the source database is constantly updated."
\end{itemize}


\subsection{The \CRANpkg{pxweb} package}

% In early 2013, Statistics Sweden released PX-WEB as a way to simplify the use and re-use of the statistics produced by the agency \cite{xxx}. The PXWEB API


\begin{itemize}
    \item History of the package
    \item Design principles
    \item (See comment out text below)
    \item Extendibility of the package with new APIS
    \item Can be used in other packages as the workhorse for accessing API data (See Oyvinds project).
\end{itemize}

% In 2018, we made major design decisions and largely rewrote the package in order to simplify the overall design while improving the overall capabilities and efficiency. Hence the current, mature version, is a result of active development and testing by the user community 

% Whereas dozens of statistical authorities have started to share data through the PX-WEB API,  a dedicated R package that provides a unified access to these data collections has been missing. 

% The \CRANpkg{pxweb} package is now filling this gap [CLOSELY RELATED PKGS SHOULD BE CITED HERE?]. Following its first CRAN release in 2014, the \CRANpkg{pxweb}, several contributors and feedback from the user community have supported the package development. 
% [HAS THE PKG BEEN APPLIED IN PUBLICATIONS. THIS WOULD BE A GOOD PLACE TO CITE THOSE? -> OR in DISCUSSION?].  SOME brief WORDS ABOUT DATA STANDARDS AND POSSIBLE VARIATIONS BETWEEN DATA PROVIDERS; further details will be in the later section. The pxweb depends on further R packages including \pkg{checkmate} \citep{checkmate}, \pkg{httr} \citep{httr}, \pkg{jsonlite} \citep{jsonlite}. The \CRANpkg{pxweb} package is part of the rOpenGov open data science project \citep{Lahti13icml}. 

% In summary, the \CRANpkg{pxweb} package provides custom tools for open statistical data resources provided through the PX-WEB API. Currently, the pxweb package provides seamless algorithmic access from the R environment to dozens of data collections from national authorities in countries such as Estonia, Iceland, Finland, Norway, Sweden, The Netherlands, and elsewhere. Seamless integration with other data analysis tools is facilitated by support for features such as cache, date formatting, tidy data principles \citep{wickham2014}, and the \Cpkg{tibble} \citep{tibble} data format. In this article, we provide an overview of the functionality and use cases based on the current CRAN release version (0.8). The comprehensive on-line documentation, which is available via the package homepage\footnote{\url{http://ropengov.github.io/pxweb}}, includes simple examples for individual functions, generic tutorials, and links to more advanced case studies. Moreover, the package is following best practices in open source software development such as version control, automated unit tests, continuous integration, and collaborative development \citep{PerezRiverol2016}.


%The work has been released as open source under the permissive
%modified BSD-2-clause
%license\footnote{\url{https://opensource.org/licenses/BSD-2-Clause}},
%which is permissive license and suited for research
%use \cite{Morin2012}. We appreciate feedback from the users through
%the Github issue
%tracker\footnote{\url{https://github.com/rOpenGov/pxweb/issues}}, or
%contributions through pull requests.

The package facilitates algorithmic access to data from national and
regional authorities in 18 countries, territories, and international
organizations, mainly from Europe. The current data catalogue provides
integrated access to 30 readily accessible databases
(Table~\ref{tab:databases}), and support for specifying additional
sources is available\footnote{Further organizations using PX-WEB are
listed in
https://www.scb.se/en/services/statistical-programs-for-px-files/px-web/pxweb-examples} %
It would be very good to systematically add in the API catalog these
and others we can find now, should be straightfwd.

\begin{table}
\include{api}
\caption{\label{tab:databases}PX-Web databases that are integrated in the pxweb R package API catalog. The online sources are listed in the pxweb R package. The language codes refer to the ISO 2 Letter Language Codes.}
\end{table}


\section[Usage]{Example case studies}

\begin{itemize}
    \item Short version of the vignette.  \item A nice figure and
    table should be the result.  \item Cover all relevant
    functionality, at least by mentioning it and citing the package
    documentation/website/vignette, if not all can be included here
\end{itemize}




\subsection{Citing data using pxweb}

\subsection{Using it for another API, not in the catalogue}


\section[summary]{Discussion}

%\begin{itemize}
%    \item Reiterate the gap that this package fills: data access for open workflows; summary of the functionality
%    \item The present version of the package is mature and stable; information on the userbase and downloads?
%    \item Quality control: CI, unit tests, open development/issues, CRAN checks etc
%    \item Justification for design choices that are potentially interesting or controversial
%    \item Examples of known case studies etc. that the package has enabled
%    \item Future extensions: additional data sources, additional functionality(?)..? 
%\end{itemize}


% Summary of the package and motivation

The \CRANpkg{pxweb} package provides a seamless programmatic access to
statistical data resources that are shared via the PX-WEB API. This is
helping to bridge the gap between the providers and end users of
official statistics. Whereas specialized web applications typically
focus on a particular data source or task \cite{xxx}, \CRANpkg{pxweb}
facilitates general programmatic access to open APIs that share data
in the PC-Axis format, which has been widely adopted by national and
international statistical organizations. A user gets a seamless and
standardized access to original online data sources, which allows the
implementation of open and reproducible data science workflows on
official statistics \citep{Gandrud13, Boettiger2015} and supports FAIR
data sharing \cite{xxx}. As such, the package solves a timely
bottleneck in governmental data analytics as the availability of open
data from National Statistical Agencies has been steadily increasing
\cite{xxx}.

% Summary of the functionality

The package facilitates algorithmic access and analysis of a
remarkable vast collection of curated data collections from the R
environment to data from national authorities in over a dozen
countries or international organizations, mainly from Europe. The data
catalogue integrated with the package lists 30 readily accessible
databases, and the methods allow the users to specify additional API
sources when necessary. The package automates major parts of the data
science workflow, such as finding, accessing, integrating, citing, and
reporting data. The available tools include utilities for data query,
download, manipulation and visualization, and they can utilize
information about the incorporated data hierarchies. The combination
of algorithms provides a smooth, automated, reproducible and
well-documented access to continuously evolving statistical data
sources. The online documentation provides detailed examples on how
the package can be used to investigate spatial, temporal, demographic,
and other phenomena. The implemented methods take into account
variations in raw data formats, access details, tidy data
principles \citep{wickham2014}, and typical use cases so that the end
users can avoid repetitive programming tasks, avoid potential
misinterpretations and coding errors. This facilitates integration
with other data analysis tools, and helps the end users to dedicate
more time on the statistical analysis and interpretation. In addition
to helping to identify and access data, the package simplifies and
standardizes the process of data citations with specific data
versions, access times, and sources. Our implementations provide
automatically collected citation information and details for the
accessed data sets and version numbers, thus facilitating transparent
and reproducible research in the ever-changing digital
landscape. Automation of the citation data collection is not only
saving time by increased efficiency but also improving the reliability
and accuracy of the citation data. By providing these tools we hope to
promote more wide-spread adoption of data citation
guidelines \cite{xxx}.

The current, mature version is a result of active development and
testing by the user community since its first CRAN release in 2014 and
a major revision in 2018. The introduced tools can benefit researchers
and data analysts particularly in academia, government, and industry,
but also citizen scientists and NGOs. We expect that the package has
been adopted especially by who are analysing official statistical data
in R and implementing their own data science workflows. The package
has a stable and thoroughly tested functionality. Following the major
rewrite of the package in 2018, the number of downloads has tripled
from 3000 downloads in 2017 to 11000 downloads in 2021. The package is
currently the second most downloaded package of the rOpenGov project
after the eurostat \CRANpkg{pxweb} package \cite{Lahti17eurostat}, and
has roughly the same number of downloads with the \CRANpkg{osmar}
package for the Open Street Map \cite{osmar}.

%Quality control: CI, unit tests, open development/issues, CRAN checks etc

The package follows best practices in open source software development
such as version control, automated unit tests, continuous integration,
and collaborative development \citep{PerezRiverol2016}. Release
through CRAN ensures compatibility with the broader R ecosystem. We
hope that our active commitment to the project maintenance and
development of the package will encourage further feedback and
contributions from the user community.

%Justification for design choices that are potentially interesting or controversial

Whereas \CRANpkg{pxweb} has been designed to access the PX-Web API,
this should not be confused with the related PC-Axis file format
(typically abbreviated as '.px'). We anticipate that the more flexible
PC Axis API is gradually taking over the PC-Axis file format as the
data sharing platform for official statistics. Those who need to
access and parse legacy px files can have a look at the independently
developed pxR package, which is currently maintained in
Github \url{https://github.com/cjgb/pxR}.

%Examples of known case studies etc. that the package has enabled

Whereas the methods can be used with any PX-WEB API that is locally
accessible, an increasing number of the official statistical resources
are open access. The statistical authorities in many nordic countries
have invested in open data sharing, which supports use cases by
governmental authorities, companies, and citizen scientists. The
package has been used, for instance, in independent studies on
electronic gambling machines and socioeconomic
status \cite{Raisamo2019}. % Didn't find other citations.

% Future extensions: additional data sources, additional functionality(?)..? 
% Currently unclear to me what is the added value in PxWebApiData / LL
% here discuss the new pkg by our collaborators -> does this refer to PxWebApiData?
% Unexpected use cases by integration with external sources -> Any ideas?

The \CRANpkg{pxweb} package has been designed to provide the core
functionality for API access, around which further custom tools and
functionality can be built. Future developments of the package will
include improved query options, analytical, and visualization
capabilities. Examples include the independently
developed \CRANpkg{PxWebApiData}, which adds specific functionality in
the nordic countries (Norway, Sweden, Finland), and the
\CRANpkg{geofi} package combines statistical information with tools for
geospatial visualization. Besides research use, official statistics
provide ample material for training in statistics as well as in
computational humanities and social sciences and other fields. Thus,
adding interactive features or specialized tools targeting selected
data sources could support pedagogical case studies.

% Concluding

Transparency and reproducibility of statistical workflows from raw
data to statistical summaries and final publication can be greatly
facilitated by combining programmatic data access with downstream data
analysis and visualization tools. The pxweb package supports
automated, transparent, reproducible, and well-documented data
retrieval from statistical authorities. Programmatic access to data
resources and the availability of well-tested downstream analysis
methods facilitates the implementation of open and reproducible data
science workflows. The \CRANpkg{pxweb} package provides improvements
over the previously available methods, and it has been extensively
tested and refined by an active user community. The work contributes
to the rapidly growing field of open data science \cite{Lahti2018IDA,
xxx} and helps to make up-to-date historical and contemporary data
collections from dozens of statistical authorities more easily
accessible by the statistical analysis research and education. This
could be anticipated to encourage further data sharing by the
authorities as the value of the data is increasing together with the
user base and the number of complementary methods, workflows, and
applications.



\section*{Acknowledgments}

We are grateful to all package contributors. The work has been
partially funded by Academy of Finland (decisions 295741, 345630 to
LL), and is part of
rOpenGov\footnote{\url{https://github.ropengov.io}}.


\bibliography{magnusson-kainu-lahti}

  

\address{M\r{a}ns Magnusson\\
  Department of Computer Science\\
  Aalto University\\
  Finland\\}
\email{mons.magnusson@gmail.com}

%\address{Markus Kainu\\
%  %Research Department, The Social Insurance Institution of Finland\\
%  %PO Box 450, 00101 Helsinki\\
%  Finland\\}
%\email{markus.kainu@kela.fi}

\address{Leo Lahti\\
  Department of Computing\\
  PO Box 20014 University of Turku\\
  Finland\\}
\email{leo.lahti@iki.fi}
